% $Id: usersch4.tex,v 1.98 2006/04/05 16:42:32 kb Exp $
% Copyright (c) 2000  The PARI Group
%
% This file is part of the PARI/GP documentation
%
% Permission is granted to copy, distribute and/or modify this document
% under the terms of the GNU General Public License
\chapter{Programming PARI in Library Mode}

\noindent The \emph{User's Guide to Pari/GP} gives in three chapters a
general presentation of the system, of the \kbd{gp} calculator, and detailed
explanation of high level PARI routines available through the calculator. The
present manual assumes general familiarity with the contents of these
chapters and the basics of ANSI C programming, and focuses on the usage of
the PARI library. In this chapter, we introduce the general concepts of PARI
programming and describe useful general purpose functions. Chapter 5
describes all available public low-level functions.

\section{Introduction: initializations, universal objects}
\label{se:intro4}

\noindent
To use PARI in \idx{library mode}, you must write a C program and link it to
the PARI library. See the installation guide or the Appendix to the
\emph{User's Guide to Pari/GP} on how to create and install the library and
include files. A sample Makefile is presented in Appendix~A, and a more
elaborate one in \kbd{examples/Makefile}. The best way to understand how
programming is done is to work through a complete example. We will write such
a program in \secref{se:prog}. Before doing this, a few explanations are in
order.

First, one must explain to the outside world what kind of objects and
routines we are going to use. This is done with the directive

\bprog
#include <pari.h>
@eprog
\noindent
In particular, this header defines the fundamental type for all PARI objects:
the type \teb{GEN}, which is simply a pointer to \kbd{long}.

Before any PARI routine is called, one must initialize the system, and in
particular the PARI stack which is both a scratchboard and a repository for
computed objects. This is done with a call to the function

\fun{void}{pari_init}{size\_t size, ulong maxprime}

\noindent The first argument is the number of bytes given to PARI to work
with, and the second is the upper limit on a precomputed prime number table;
\kbd{size} should not reasonably be taken below $500000$ but you may set
$\tet{maxprime} = 0$, although the system still needs to precompute all
primes up to about $2^{16}$.

\noindent We have now at our disposal:

$\bullet$ a PARI \tev{stack} containing nothing. It is a big
connected chunk of \kbd{size} bytes of memory. All your computations
take place here. In large computations, unwanted intermediate results quickly
clutter up memory so some kind of garbage collecting is needed. Most large
systems do garbage collecting when the memory is getting scarce, and this
slows down the performance. PARI takes a different approach: you must do your
own cleaning up when the intermediate results are not needed anymore. Special
purpose routines have been written to do this; we will see later how (and
when) you should use them.

$\bullet$ the following \emph{universal objects} (by definition, objects
which do not belong to the stack): the integers $0$, $1$, $-1$ and $2$
(respectively called \tet{gen_0}, \tet{gen_1}, \tet{gen_m1} and
\tet{gen_2}), the fraction $\dfrac{1}{2}$ (\tet{ghalf}), the complex number
$i$ (\tet{gi}). All of these are of type \kbd{GEN}.

In addition, space is reserved for the polynomials $x_v$
\sidx{variable}
(\tet{pol_x}\kbd{[$v$]}), and the polynomials $1_v$ (\tet{pol_1}\kbd{[$v$]}).
Here, $x_v$ is the name of variable number $v$, where $0\le v\le
\tet{MAXVARN}$. Both \tet{pol_1} and \tet{pol_x} are arrays of \kbd{GEN}s, the
index being the polynomial variable number.

However, except for the ones corresponding to variables $0$ and \kbd{MAXVARN},
these polynomials are \emph{not} created upon initialization. It
is the programmer's responsibility to fill them before use. We will see how
this is done in \secref{se:vars} (\emph{never} through direct assignment).

$\bullet$ a \tev{heap} which is just a linked list of permanent
universal objects. For now, it contains exactly the ones listed above. You
will probably very rarely use the heap yourself; and if so, only as a
collection of copies of objects taken from the stack (called \idx{clone}s in
the sequel). Thus you need not bother with its internal structure, which may
change as PARI evolves. Some complex PARI functions create clones for special
garbage collecting purposes, usually destroying them when returning.

$\bullet$ a table of primes (in fact of \emph{differences} between
consecutive primes), called \teb{diffptr}, of type \kbd{byteptr}
(pointer to \kbd{unsigned char}). Its use is described in appendix~B.

$\bullet$ access to all the built-in functions of the PARI library.
These are declared to the outside world when you include \kbd{pari.h}, but
need the above things to function properly. So if you forget the call to
\tet{pari_init}, you will get a fatal error when running your program.

\section{Important technical notes}

\subsec{Types}.

\noindent
Although PARI objects all have the C type \kbd{GEN}, we will freely use
the word \teb{type} to refer to PARI dynamic subtypes: \typ{INT}, \typ{REAL},
etc. The declaration
\bprog
  GEN x;
@eprog\noindent
declares a C variable of type \kbd{GEN}, but its ``value'' will be said to
have type \typ{INT}, \typ{REAL}, etc. The meaning should always be clear from
the context.

\subsec{Type recursivity}.

\noindent
Conceptually, most PARI types are recursive. But the \kbd{GEN} type is a
pointer to \kbd{long}, not to \kbd{GEN}. So special macros must be used to
access \kbd{GEN}'s components. The simplest one is \tet{gel}$(V, i)$, where
\key{el} stands for \key{el}ement, to access component number $i$ of the
\kbd{GEN} $V$. This is a valid \kbd{lvalue} (may be put on the left side of
an assignment), and the following two constructions are exceedingly frequent
%
\bprog
  gel(V, i) = x;
  x = gel(V, i);
@eprog\noindent
where \kbd{x} and \kbd{V} are \kbd{GEN}s. This macro accesses and modifies
directly the components of $V$ and do not create a copy of the coefficient,
contrary to all the library \emph{functions}.

More generally, to retrieve the values of elements of lists of \dots\ of
lists of vectors we have the \tet{gmael} macros (for {\bf m}ultidimensional
{\bf a}rray {\bf el}ement). The syntax is $\key{gmael}n(V,a_1,\dots,a_n)$,
where $V$ is a \kbd{GEN}, the $a_i$ are indexes, and $n$ is an integer
between $1$ and $5$. This stands for $x[a_1][a_2]\dots[a_n]$, and returns a
\kbd{GEN}. The macros \tet{gel} (resp.~\tet{gmael}) are synonyms for
\tet{gmael1} (resp.~\kbd{gmael2}).

Finally, the macro $\tet{gcoeff}(M, i, j)$ has exactly the meaning of
\kbd{M[i,j]} in GP when \kbd{M} is a matrix. Note that due to the
implementation of \typ{MAT}s as horizontal lists of vertical vectors,
\kbd{gcoeff(x,y)} is actually equivalent to \kbd{gmael(y,x)}. One should use
\kbd{gcoeff} in matrix context, and \kbd{gmael} otherwise.

\subsec{Variations on basic functions}.\label{se:low_level} In the library
syntax descriptions in Chapter~3, we have only given the basic names of the
functions. For example \kbd{gadd}$(x,y)$ assumes that $x$ and $y$ are
\kbd{GEN}s, and \emph{creates} the result $x+y$ on the PARI stack. For most
of the basic operators and functions, many other variants are available. We
give some examples for \kbd{gadd}, but the same is true for all the basic
operators, as well as for some simple common functions (a complete list
is given in Chapter~5):

\fun{GEN}{gaddgs}{GEN x, long y}

\fun{GEN}{gaddsg}{long x, GEN y}

\noindent In the following three, \kbd{z} is a preexisting \kbd{GEN} and the
result of the corresponding operation is put into~\kbd{z}. The size of the PARI
stack does not change:

\fun{void}{gaddz}{GEN x, GEN y, GEN z}

\fun{void}{gaddgsz}{GEN x, long y, GEN z}

\fun{void}{gaddsgz}{GEN x, GEN y, GEN z}

\noindent There are also low level functions which are special cases of the
above:

\fun{GEN}{addii}{GEN x, GEN y}: here $x$ and $y$ are \kbd{GEN}s of type
\typ{INT} (this is not checked).

\fun{GEN}{addrr}{GEN x, GEN y}: here $x$ and $y$ are \kbd{GEN}s of
type \typ{REAL} (this is not checked).

\noindent
There also exist functions \teb{addir}, \teb{addri}, \teb{mpadd} (whose
two arguments can be of type \typ{INT} or \typ{REAL}), \teb{addis} (to add a
\typ{INT} and a \kbd{long}) and so on.

All these specialized functions are of course more efficient than the general
purpose ones, but note the hidden danger here: the types of the objects
involved, if they are themselves results of a previous computation, are not
completely predetermined. For instance the multiplication of a \typ{REAL} by
a \typ{INT} \emph{usually} gives a \typ{REAL} result, except when the integer
is~0, in which case according to the PARI philosophy the result is the exact
integer~0. Hence if afterwards you call a function which specifically needs a
\typ{REAL} argument, you are in trouble.

The names are self-explanatory once you know that {\bf i} stands for a
\typ{INT}, {\bf r} for a \typ{REAL}, {\bf mp} for i or r, {\bf s} for a
signed C long integer, {\bf u} for an unsigned C long integer; finally the
suffix {\bf z} means that the result is not created on the PARI stack but
assigned to a preexisting GEN object passed as an extra argument. For
completeness, Chapter 5 gives a description of these low-level functions.

\subsec{Portability: 32-bit / 64-bit architectures}.

\noindent
PARI supports both 32-bit and 64-bit based machines, but not simultaneously!
The library will have been compiled assuming a given architecture, and some
of the header files you include (through \kbd{pari.h}) will have been
modified to match the library.

Portable macros are defined to bypass most machine dependencies. If you want
your programs to run identically on 32-bit and 64-bit machines, you have to
use these, and not the corresponding numeric values, whenever the precise
size of your \kbd{long} integers might matter. Here are the most important
ones:

\settabs\+ -----------------------------&---------------&------------&\cr \+
& 64-bit  & 32-bit \cr\+ \tet{BITS_IN_LONG}  & 64      & 32 \cr\+
\tet{LONG_IS_64BIT} & defined & undefined \cr\+
\tet{DEFAULTPREC}   & 3       & 4 & ($\approx$ 19 decimal digits, %
see formula below) \cr\+
\tet{MEDDEFAULTPREC}& 4       & 6 & ($\approx$ 38 decimal digits) \cr\+
\tet{BIGDEFAULTPREC}& 5       & 8 & ($\approx$ 57 decimal digits) \cr
\noindent For instance, suppose you call a transcendental function, such as

\kbd{GEN \key{gexp}(GEN x, long prec)}.

\noindent The last argument \kbd{prec} is only used if \kbd{x} is an exact
object, otherwise the relative precision is determined by the precision
of~\kbd{x}. But since \kbd{prec} sets the size of the inexact result counted
in (\kbd{long}) \emph{words} (including codewords), the same value of
\kbd{prec} will yield different results on 32-bit and 64-bit machines. Real
numbers have two codewords (see~\secref{se:impl}), so the formula for
computing the bit accuracy is $$ \tet{bit_accuracy}(\kbd{prec}) = (\kbd{prec}
- 2) * \tet{BITS_IN_LONG}$$ (this is actually the definition of a macro). The
  corresponding accuracy expressed in decimal digits would be
%
$$ \kbd{bit\_accuracy(prec)} * \log(2) / \log(10).$$
%
For example if the value of \kbd{prec} is 5, the corresponding accuracy for
32-bit machines is $(5-2)*\log(2^{32})/\log(10)\approx 28$ decimal digits,
while for 64-bit machines it is $(5-2)*\log(2^{64})/\log(10)\approx 57$
decimal digits.

Thus, you must take care to change the \kbd{prec} parameter you are supplying
according to the bit size, either using the default precisions given by the
various \kbd{DEFAULTPREC}s, or by using conditional constructs of the form:
%
\bprog
#ifndef LONG_IS_64BIT
  prec = 4;
#else
  prec = 6;
#endif
@eprog
\noindent which is in this case equivalent to the statement
\kbd{prec = MEDDEFAULTPREC;}.

Note that for parity reasons, half the accuracies available on 32-bit
architectures (the odd ones) have no precise equivalents on 64-bit machines.

\section{Garbage collection}\label{se:garbage}\sidx{garbage collecting}

\subsec{Why and how}.

\noindent
As we have seen, the \kbd{pari\_init} routine allocates a big range of
addresses, the \tev{stack}, that are going to be used throughout. Recall
that all PARI objects are pointers. Except for a few universal objects,
they all point at some part of the stack.

The stack starts at the address \kbd{bot} and ends just before \kbd{top}. This
means that the quantity
%
$$ (\kbd{top} - \kbd{bot})\,/\,\kbd{sizeof(long)} $$
%
is (roughly) equal to the \kbd{size} argument of \kbd{pari\_init}. The PARI
stack also has a ``current stack pointer'' called \teb{avma}, which stands
for {\bf av}ailable {\bf m}emory {\bf a}ddress. These three variables are
global (declared by \kbd{pari.h}). They are of type \tet{pari_sp}, which
means \emph{pari stack pointer}.

The stack is oriented upside-down: the more recent an object, the closer to
\kbd{bot}. Accordingly, initially \kbd{avma} = \kbd{top}, and \kbd{avma} gets
\emph{decremented} as new objects are created. As its name indicates,
\kbd{avma} always points just \emph{after} the first free address on the
stack, and \kbd{(GEN)avma} is always (a pointer to) the latest created object.
When \kbd{avma} reaches \kbd{bot}, the stack overflows, aborting all
computations, and an error message is issued. To avoid this \emph{you}
need to clean up the stack from time to time, when intermediate objects are
not needed anymore. This is called ``\emph{garbage collecting}.''

We are now going to describe briefly how this is done. We will see many
concrete examples in the next subsection.

\noindent$\bullet$
First, PARI routines do their own garbage collecting, which means that
whenever a documented function from the library returns, only its result(s)
have been added to the stack (non-documented ones may not do this). In
particular, a PARI function that does not return a \kbd{GEN} does not clutter
the stack. Thus, if your computation is small enough (e.g.~you call few PARI
routines, or most of them return \kbd{long} integers), then you do not need
to do any garbage collecting. This is probably the case in many of your
subroutines. Of course the objects that were on the stack \emph{before} the
function call are left alone. Except for the ones listed below, PARI
functions only collect their own garbage.

\noindent$\bullet$
It may happen that all objects that were created after a certain point can
be deleted~--- for instance, if the final result you need is not a
\kbd{GEN}, or if some search proved futile. Then, it is enough to record
the value of \kbd{avma} just \emph{before} the first garbage is created,
and restore it upon exit:

\bprog
pari_sp av = avma; /*@Ccom record initial avma */

garbage ...
avma = av; /*@Ccom restore it */
@eprog
\noindent All objects created in the \kbd{garbage} zone will eventually
be overwritten: they should not be accessed anymore once \kbd{avma} has been
restored.

\noindent$\bullet$
If you want to destroy (i.e.~give back the memory occupied by) the
\emph{latest} PARI object on the stack (e.g.~the latest one obtained from a
function call), you can use the function\sidx{destruction}%
\vadjust{\penalty500}%discourage page break

\fun{void}{cgiv}{GEN z}

\noindent where \kbd{z} is the object you want to give back. This is
equivalent to the above where the initial \kbd{av} is computed from \kbd{z}.

\noindent$\bullet$
Unfortunately life is not so simple, and sometimes you will want
to give back accumulated garbage \emph{during} a computation without losing
recent data. For this you need the \kbd{gerepile} function (or one of its
simpler variants described hereafter):

\fun{GEN}{gerepile}{pari\_sp ltop, pari\_sp lbot, GEN q}

\noindent
This function cleans up the stack between \kbd{ltop} and \kbd{lbot}, where
$\kbd{lbot} < \kbd{ltop}$, and returns the updated object \kbd{q}. This means:

1) we translate (copy) all the objects in the interval
$[\kbd{avma}, \kbd{lbot}[$, so that its right extremity abuts the address
\kbd{ltop}. Graphically

\vbox{\bprog
             bot           avma   lbot          ltop     top
End of stack  |-------------[++++++[-/-/-/-/-/-/-|++++++++|  Start
                free memory            garbage
@eprog
\noindent becomes:
\bprog
             bot                         avma   ltop     top
End of stack  |---------------------------[++++++[++++++++|  Start
                       free memory
@eprog
}
\noindent where \kbd{++} denote significant objects, \kbd{--} the unused part
of the stack, and \kbd{-/-} the garbage we remove.

2) The function then inspects all the PARI objects between \kbd{avma} and
\kbd{lbot} (i.e.~the ones that we want to keep and that have been translated)
and looks at every component of such an object which is not a codeword. Each
such component is a pointer to an object whose address is either

--- between \kbd{avma} and \kbd{lbot}, in which case it is suitably updated,

--- larger than or equal to \kbd{ltop}, in which case it does not change, or

--- between \kbd{lbot} and \kbd{ltop} in which case \kbd{gerepile}
raises an error (``significant pointers lost in gerepile'').

3) \key{avma} is updated (we add $\kbd{ltop} - \kbd{lbot}$ to the old value).

4) We return the (possibly updated) object \kbd{q}: if \kbd{q} initially
pointed between \kbd{avma} and \kbd{lbot}, we return the updated address, as
in~2). If not, the original address is still valid, and is returned!

As stated above, no component of the remaining objects (in particular
\kbd{q}) should belong to the erased segment [\kbd{lbot}, \kbd{ltop}[, and
this is checked within \kbd{gerepile}. But beware as well that the addresses
of the objects in the translated zone change after a call to \kbd{gerepile},
so you must not access any pointer which previously pointed into the zone
below \kbd{ltop}. If you need to recover more than one object, use one of the
\kbd{gerepilemany} functions below.

As a consequence of the preceding explanation, if a PARI object is to be
relocated by \hbox{gerepile} then, apart from universal objects, the chunks
of memory used by its components should be in consecutive memory locations.
All \kbd{GEN}s created by documented PARI functions are guaranteed to satisfy
this. This is because the \kbd{gerepile} function knows only about \emph{two
connected zones}: the garbage that is erased (between \kbd{lbot} and
\kbd{ltop}) and the significant pointers that are copied and updated. If
there is garbage interspersed with your objects, disaster occurs when we try
to update them and consider the corresponding ``pointers''. In most cases of
course the said garbage is in fact a bunch of other \kbd{GEN}s, in which case
we simply waste time copying and updating them for nothing. But be wary when
you allow objects to become disconnected.

\noindent In practice this is achieved by the following programming idiom:
\bprog
  ltop = avma; garbage(); lbot = avma; q = anything();
  return gerepile(ltop, lbot, q); /*@Ccom returns the updated q */
@eprog

\noindent Beware that
\bprog
  ltop = avma; garbage();
  return gerepile(ltop, avma, anything())
@eprog

\noindent might work, but should be frowned upon. We cannot predict whether
\kbd{avma} is evaluated after or before the call to \kbd{anything()}: it
depends on the compiler. If we are out of luck, it is \emph{after} the
call, so the result belongs to the garbage zone and the \kbd{gerepile}
statement becomes equivalent to \kbd{avma = ltop}. Thus we return a
pointer to random garbage.

\noindent$\bullet$ A simple variant is

\fun{GEN}{gerepileupto}{pari\_sp ltop, GEN q}

\noindent which cleans the stack between \kbd{ltop} and the \emph{connected}
object \kbd{q} and returns \kbd{q} updated. For this to work, \kbd{q} must
have been created \emph{before} all its components, otherwise they would
belong to the garbage zone! Unless mentioned otherwise, documented PARI
functions guarantee this.

\noindent$\bullet$ Another variant (a special case of \tet{gerepileall}
below, where $n=1$) is

\fun{GEN}{gerepilecopy}{pari\_sp ltop, GEN x)}

\noindent which is functionally equivalent to \kbd{gerepileupto(ltop,
gcopy(x))} but more efficient. In this case, the \kbd{GEN} parameter \kbd{x}
need not satisfy any property before the garbage collection (it may be
disconnected, components created before the root and so on). Of course, this
is less efficient than either \tet{gerepileupto} or \tet{gerepile}, because
\kbd{x} has to be copied to a clean stack zone first.

\noindent$\bullet$
To cope with complicated cases where many objects have to be
preserved, you can use

\fun{void}{gerepileall}{pari\_sp ltop, int n, ...}

\noindent where the routine expects $n$ further arguments, which are the
\emph{addresses} of the \kbd{GEN}s you want to preserve. It cleans up the most
recent part of the stack (between \kbd{ltop} and \kbd{avma}), updating all
the \kbd{GEN}s added to the argument list. A copy is done just before the
cleaning to preserve them, so they do not need to be connected before the
call. With \kbd{gerepilecopy}, this is the most robust of the \kbd{gerepile}
functions (the less prone to user error), hence the slowest.

An alternative syntax, obsolete but kept for backward compatibility, is
given by

\fun{void}{gerepilemany}{pari\_sp ltop, GEN *gptr[], int n}

\noindent which works exactly as above, except that the preserved \kbd{GEN}s
are the elements of the array \kbd{gptr} (of length $n$): \kbd{gptr[0]},
\kbd{gptr[1]}, \dots, \kbd{gptr[$n$-1]}.

\noindent$\bullet$ More efficient, but tricky to use is

\fun{void}{gerepilemanysp}{pari\_sp ltop, pari\_sp lbot, GEN *gptr[], int n}

\noindent which cleans the stack between \kbd{lbot} and \kbd{ltop} and
updates the \kbd{GEN}s pointed at by the elements of \kbd{gptr} without doing
any copying. This is subject to the same restrictions as \kbd{gerepile}, the
only difference being that more than one address gets updated.

\subsec{Examples}.

\subsubsec{gerepile}

Let \kbd{x} and \kbd{y} be two preexisting PARI objects and suppose that we
want to compute $\kbd{x}^2 + \kbd{y}^2$. This is done using the following
program:
\bprog
  GEN p1 = gsqr(x);
  GEN p2 = gsqr(y), z = gadd(p1,p2);
@eprog\noindent
The \kbd{GEN} \kbd{z} indeed points at the desired quantity. However,
consider the stack: it contains as unnecessary garbage \kbd{p1} and \kbd{p2}.
More precisely it contains (in this order) \kbd{z}, \kbd{p2}, \kbd{p1}.
(Recall that, since the stack grows downward from the top, the most recent
object comes first.)

It is not possible to get rid of \kbd{p1}, \kbd{p2} before \kbd{z} is
computed, since they are used in the final operation. We cannot record
\kbd{avma} before \kbd{p1} is computed and restore it later, since this would
destroy \kbd{z} as well. It is not possible either to use the function
\kbd{cgiv} since \kbd{p1} and \kbd{p2} are not at the bottom of the stack and
we do not want to give back~\kbd{z}.

But using \kbd{gerepile}, we can give back the memory locations corresponding
to \kbd{p1}, \kbd{p2}, and move the object \kbd{z} upwards so that no
space is lost. Specifically:
\bprog
  pari_sp ltop = avma;  /*@Ccom remember the current address of the top of the stack */
  GEN p1 = gsqr(x);
  GEN p2 = gsqr(y);
  pari_sp lbot = avma;  /*@Ccom keep the address of the bottom of the garbage pile */
  GEN z = gadd(p1, p2); /*@Ccom z is now the last object on the stack */
  z = gerepile(ltop, lbot, z);
@eprog
\noindent Of course, the last two instructions could also have been
written more simply:
\bprog
  z = gerepile(ltop, lbot, gadd(p1,p2));
@eprog\noindent In fact \kbd{gerepileupto} is even simpler to use, because
the result of \kbd{gadd} is the last object on the stack and \kbd{gadd}
is guaranteed to return an object suitable for \kbd{gerepileupto}:
\bprog
  ltop = avma;
  z = gerepileupto(ltop, gadd(gsqr(x), gsqr(y)));
@eprog\noindent
Make sure you understand exactly what has happened before you go on (use the
figure from the preceding section).

\misctitle{Remark on assignments and gerepile}: When the tree structure and
the size of the PARI objects which will appear in a computation are under
control, one may allocate sufficiently large objects at the beginning,
use assignment statements, then simply restore \kbd{avma}. Coming back to the
above example, note that \emph{if} we know that x and y are of type real
fitting into \kbd{DEFAULTPREC} words, we can program without using
\kbd{gerepile} at all:
\bprog
  z = cgetr(DEFAULTPREC); ltop = avma;
  gaffect(gadd(gsqr(x), gsqr(y)), z);
  avma = ltop;
@eprog\noindent This is often \emph{slower} than a craftily used
\kbd{gerepile} though, and certainly more cumbersome to use. As a rule,
assignment statements should generally be avoided.

\misctitle{Variations on a theme}: it is often necessary to do several
\kbd{gerepile}s during a computation. However, the fewer the better. The only
condition for \kbd{gerepile} to work is that the garbage be connected. If the
computation can be arranged so that there is a minimal number of connected
pieces of garbage, then it should be done that way.

For example suppose we want to write a function of two \kbd{GEN} variables
\kbd{x} and \kbd{y} which creates the vector $\kbd{[x}^2+\kbd{y},
\kbd{y}^2+\kbd{x]}$. Without garbage collecting, one would write:
%
\bprog
  p1 = gsqr(x); p2 = gadd(p1, y);
  p3 = gsqr(y); p4 = gadd(p3, x); z = cgetg(3, t_VEC);
  gel(z, 1) = p2;
  gel(z, 2) = p4;
@eprog\noindent
This leaves a dirty stack containing (in this order) \kbd{z}, \kbd{p4},
\kbd{p3}, \kbd{p2}, \kbd{p1}. The garbage here consists of \kbd{p1} and
\kbd{p3}, which are separated by \kbd{p2}. But if we compute \kbd{p3}
\emph{before} \kbd{p2} then the garbage becomes connected, and we get the
following program with garbage collecting:
%
\bprog
  ltop = avma; p1 = gsqr(x); p3 = gsqr(y);
  lbot = avma; z = cgetg(3, t_VEC);
  gel(z, 1) = gadd(p1,y);
  gel(z, 2) = gadd(p3,x); z = gerepile(ltop,lbot,z);
@eprog\noindent Finishing by \kbd{z = gerepileupto(ltop, z)} would be ok as
well. Beware that
\bprog
  ltop = avma; p1 = gadd(gsqr(x), y); p3 = gadd(gsqr(y), x);
  z = cgetg(3, t_VEC);
  gel(z, 1) = p1;
  gel(z, 2) = p3; z = gerepileupto(ltop,z); /*@Ccom WRONG */
@eprog\noindent
is a disaster since \kbd{p1} and \kbd{p3} are created before
\kbd{z}, so the call to \kbd{gerepileupto} overwrites them, leaving
\kbd{gel(z, 1)} and \kbd{gel(z, 2)} pointing at random data! On the other
hand
\bprog
  ltop = avma; z = cgetg(3, t_VEC);
  gel(z, 1) = gadd(gsqr(x), y);
  gel(z, 2) = gadd(gsqr(y), x); z = gerepileupto(ltop,z); /*@Ccom INEFFICIENT */
@eprog\noindent
leaves the results of \kbd{gsqr(x)} and \kbd{gsqr(y)} on the stack (and
lets \kbd{gerepileupto} update them for naught). Finally, the most elegant
and efficient version (with respect to time and memory use) is as follows
\bprog
  z = cgetg(3, t_VEC);
  ltop = avma; gel(z, 1) = gerepileupto(ltop, gadd(gsqr(x), y));
  ltop = avma; gel(z, 2) = gerepileupto(ltop, gadd(gsqr(y), x));
@eprog\noindent
which avoids updating the container \kbd{z} and cleans up its components
individually, as soon as they are computed.

\misctitle{One last example}. Let us compute the product of two complex
numbers $x$ and $y$, using the $3M$ method which requires 3 multiplications
instead of the obvious 4. Let $z = x*y$, and set $x = x_r + i*x_i$ and
similarly for $y$ and $z$. We compute $p_1 = x_r*y_r$, $p_2=x_i*y_i$,
$p_3=(x_r+x_i)*(y_r+y_i)$, and then we have $z_r=p_1-p_2$,
$z_i=p_3-(p_1+p_2)$. The program is as follows:
%
\bprog
ltop = avma;
p1 = gmul(gel(x,1), gel(y,1));
p2 = gmul(gel(x,2), gel(y,2));
p3 = gmul(gadd(gel(x,1), gel(x,2)), gadd(gel(y,1), gel(y,2)));
p4 = gadd(p1,p2);
lbot = avma; z = cgetg(3, t_COMPLEX);
gel(z, 1) = gsub(p1,p2);
gel(z, 2) = gsub(p3,p4); z = gerepile(ltop,lbot,z);
@eprog

\misctitle{Exercise}. Write a function which multiplies a matrix by a column
vector. Hint: start with a \kbd{cgetg} of the result, and use \kbd{gerepile}
whenever a coefficient of the result vector is computed. You can look at the
answer in \kbd{src/basemath/gen1.c:MC\_mul()}.

\subsubsec{gerepileall}

Let us now see why we may need the \kbd{gerepileall} variants. Although it
is not an infrequent occurrence, we do not give a specific example but a
general one: suppose that we want to do a computation (usually inside a
larger function) producing more than one PARI object as a result, say two for
instance. Then even if we set up the work properly, before cleaning up we
have a stack which has the desired results \kbd{z1}, \kbd{z2} (say), and
then connected garbage from lbot to ltop. If we write
\bprog
  z1 = gerepile(ltop, lbot, z1);
@eprog\noindent
then the stack is cleaned, the pointers fixed up, but we have lost the
address of \kbd{z2}. This is where we need the \idx{gerepileall}
function:
\bprog
  gerepileall(ltop, 2, &z1, &z2)
@eprog
\noindent copies \kbd{z1} and \kbd{z2} to new locations, cleans the stack
from \kbd{ltop} to the old \kbd{avma}, and updates the pointers \kbd{z1} and
\kbd{z2}. Here we do not assume anything about the stack: the garbage can be
disconnected and \kbd{z1}, \kbd{z2} need not be at the bottom of the stack.
If all of these assumptions are in fact satisfied, then we can call
\kbd{gerepilemanysp} instead, which is usually faster since we do not need
the initial copy (on the other hand, it is less cache friendly).

A most important usage is ``random'' garbage collection during loops
whose size requirements we cannot (or do not bother to) control in advance:
\bprog
  pari_sp ltop = avma, limit = stack_lim(avma, 1);
  GEN x, y;
  while (...)
  {
    garbage(); x = anything();
    garbage(); y = anything(); garbage();
    if (avma < limit) /*@Ccom memory is running low (half spent since entry) */
      gerepileall(ltop, 2, &x, &y);
  }
@eprog
\noindent Here we assume that only \kbd{x} and \kbd{y} are needed from one
iteration to the next. As it would be costly to call gerepile once for each
iteration, we only do it when it seems to have become necessary. The macro
\tet{stack_lim}\kbd{(avma,$n$)} denotes an address where $2^{n-1} /
(2^{n-1}+1)$ of the remaining stack space is exhausted ($1/2$ for $n=1$,
$2/3$ for $n=2$).

\subsec{Comments}.

First, \kbd{gerepile} has turned out to be a flexible and fast garbage
collector for number-theoretic computations, which compares favorably with
more sophisticated methods used in other systems. Our benchmarks indicate
that the price paid for using \kbd{gerepile} and \kbd{gerepile}-related
copies, when properly used, is usually less than 1 percent of the total
running time, which is quite acceptable!

Second, it is of course harder on the programmer, and quite error-prone
if you do not stick to a consistent PARI programming style. If all seems
lost, just use \tet{gerepilecopy} (or \tet{gerepileall}) to fix up the stack
for you. You can always optimize later when you have sorted out exactly which
routines are crucial and what objects need to be preserved and their usual
sizes.

\smallskip If you followed us this far, congratulations, and rejoice: the
rest is much easier.

\section{Creation of PARI objects, assignments, conversions}

\subsec{Creation of PARI objects}.\sidx{creation}
The basic function which creates a PARI object is the function
\teb{cgetg} whose prototype is:

\kbd{GEN \key{cgetg}(long length, long type)}.

\noindent
Here \kbd{length} specifies the number of longwords to be allocated to the
object, and type is the type number of the object, preferably in symbolic
form (see \secref{se:impl} for the list of these). The precise effect of
this function is as follows: it first creates on the PARI \emph{stack} a
chunk of memory of size \kbd{length} longwords, and saves the address of the
chunk which it will in the end return. If the stack has been used up, a
message to the effect that ``the PARI stack overflows'' is printed,
and an error raised. Otherwise, it sets the type and length of the PARI object.
In effect, it fills its first codeword (\kbd{z[0]} or \kbd{*z}). Many PARI
objects also have a second codeword (types \typ{INT}, \typ{REAL},
\typ{PADIC}, \typ{POL}, and \typ{SER}). In case you want to produce one of
those from scratch, which should be exceedingly rare, \emph{it is your
responsibility to fill this second codeword}, either explicitly (using the
macros described in \secref{se:impl}), or implicitly using an assignment
statement (using \kbd{gaffect}).

Note that the argument \kbd{length} is predetermined for a number of types:
3 for types \typ{INTMOD}, \typ{FRAC}, \typ{COMPLEX}, \typ{POLMOD},
\typ{RFRAC}, 4 for type \typ{QUAD} and \typ{QFI}, and 5 for type \typ{PADIC}
and \typ{QFR}. However for the sake of efficiency, no checking is done in the
function \kbd{cgetg}, so disasters will occur if you give an incorrect
length.

\misctitle{Notes}: 1)  The main use of this function is create efficiently
a constant object, or to prepare for later assignments (see
\secref{se:assign}). Most of the time you will use \kbd{GEN} objects as they
are created and returned by PARI functions. In this case you do not need to
use \kbd{cgetg} to create space to hold them.

\noindent 2) For the creation of leaves, i.e.~\typ{INT} or \typ{REAL},

\fun{GEN}{cgeti}{long length}

\fun{GEN}{cgetr}{long length}

\noindent should be used instead of \kbd{cgetg(length, t\_INT)} and
\kbd{cgetg(length, t\_REAL)} respectively. Finally

\fun{GEN}{cgetc}{long prec}

\noindent creates a \typ{COMPLEX} whose real and imaginary part are
\typ{REAL}s allocated by \kbd{cgetr(prec)}.

\misctitle{Examples}: 1) Both \kbd{z = cgeti(DEFAULTPREC)} and
\kbd{cgetg(DEFAULTPREC, t\_INT)} create a \typ{INT} whose ``precision'' is
\kbd{bit\_accuracy(DEFAULTPREC)} = 64. This means \kbd{z} can hold rational
integers of absolute value less than $2^{64}$. Note that in both cases, the
second codeword is \emph{not} filled. Of course we could use numerical
values, e.g.~\kbd{cgeti(4)}, but this would have different meanings on
different machines as \kbd{bit\_accuracy(4)} equals 64 on 32-bit machines,
but 128 on 64-bit machines.

\noindent 2) The following creates a \emph{complex number} whose real and
imaginary parts can hold real numbers of precision
$\kbd{bit\_accuracy(MEDDEFAULTPREC)} = 96\hbox{ bits:}$
%
\bprog
  z = cgetg(3, t_COMPLEX);
  gel(z, 1) = cgetr(MEDDEFAULTPREC);
  gel(z, 2) = cgetr(MEDDEFAULTPREC);
@eprog\noindent
or simply \kbd{z = cgetc(MEDDEFAULTPREC)}.

\noindent 3) To create a matrix object for $4\times 3$ matrices:
%
\bprog
  z = cgetg(4, t_MAT);
  for(i=1; i<4; i++) gel(z, i) = cgetg(5, t_COL);
@eprog
%
\noindent If one wishes to create space for the matrix elements themselves,
one has to follow this with a double loop to fill each column vector.

These last two examples illustrate the fact that since PARI types are
recursive, all the branches of the tree must be created. The function
\teb{cgetg} creates only the ``root'', and other calls to \kbd{cgetg} must be
made to produce the whole tree. For matrices, a common mistake is to think
that \kbd{z = cgetg(4, t\_MAT)} (for example) creates the root of the
matrix: one needs also to create the column vectors of the matrix (obviously,
since we specified only one dimension in the first \kbd{cgetg}!). This is
because a matrix is really just a row vector of column vectors (hence a
priori not a basic type), but it has been given a special type number so that
operations with matrices become possible.

Finally, to facilitate input of constant objects when speed is not paramount,
there are four \tet{varargs} functions:

\fun{GEN}{mkintn}{long n, ...}
returns the non-negative \typ{INT} whose development in base $2^{32}$
is given by the following $n$ words (\kbd{unsigned long}). It is assumed that
all such arguments are less than $2^{32}$ (the actual \kbd{sizeof(long)} is
irrelevant, the behaviour is also as above on $64$-bit machines).
\bprog
  mkintn(3, a2, a1, a0);
@eprog
\noindent returns $a_2 2^{64} + a_1 2^{32} + a_0$.

\fun{GEN}{mkpoln}{long n, ...}
Returns the \typ{POL} whose $n$ coefficients (\kbd{GEN}) follow, in order of
decreasing degree.
\bprog
  mkpoln(3, gen_1, gen_2, gen_0);
@eprog
\noindent returns the polynomial $X^2 + 2X$ (in variable $0$, use
\tet{setvarn} if you want other variable numbers). Beware that $n$ is the
number of coefficients, hence \emph{one more} than the degree.

\fun{GEN}{mkvecn}{long n, ...}
returns the \typ{VEC} whose $n$ coefficients (\kbd{GEN}) follow.

\fun{GEN}{mkcoln}{long n, ...}
returns the \typ{COL} whose $n$ coefficients (\kbd{GEN}) follow.

\misctitle{Warning}: Contrary to the policy of general PARI functions, the
latter three functions do \emph{not} copy their arguments, nor do they produce
an object a priori suitable for \tet{gerepileupto}. For instance
\bprog
  /*@Ccom gerepile-safe: components are universal objects */
  z = mkvecn(3, gen_1, gen_0, gen_2);

  /*@Ccom not OK for gerepileupto: stoi(3) creates component before root */
  z = mkvecn(3, stoi(3), gen_0, gen_2);

  /*@Ccom NO! First vector component \kbd{x} is destroyed */
  x = gclone(gen_1);
  z = mkvecn(3, x, gen_0, gen_2);
  gunclone(x);
@eprog

\noindent The following function is also available as a special case of
\tet{mkintn}:

\fun{GEN}{u2toi}{ulong a, ulong b}

Returns the \kbd{GEN} equal to $2^{32} a + b$, \emph{assuming} that
$a,b < 2^{32}$. This does not depend on \kbd{sizeof(long)}: the behaviour is
as above on both $32$ and $64$-bit machines.

\subsec{Assignments}.
Firstly, if \kbd{x} and \kbd{y} are both declared as \kbd{GEN} (i.e.~pointers
to something), the ordinary C assignment \kbd{y = x} makes perfect sense: we
are just moving a pointer around. However, physically modifying either
\kbd{x} or \kbd{y} (for instance, \kbd{x[1] = 0}) also changes the other
one, which is usually not desirable. \label{se:assign}

\misctitle{Very important note}: Using the functions described in this
paragraph is inefficient and often awkward: one of the \tet{gerepile}
functions (see~\secref{se:garbage}) should be preferred. See the paragraph
end for one exception to this rule.

\noindent
The general PARI \idx{assignment} function is the function \teb{gaffect} with
the following syntax:

\fun{void}{gaffect}{GEN x, GEN y}

\noindent
Its effect is to assign the PARI object \kbd{x} into the \emph{preexisting}
object \kbd{y}. This copies the whole structure of \kbd{x} into \kbd{y} so
many conditions must be met for the assignment to be possible. For instance
it is allowed to assign a \typ{INT} into a \typ{REAL}, but the converse is
forbidden. For that, you must use the truncation or rounding function of
your choice (see section 3.2). It can also happen that \kbd{y} is not large
enough or does not have the proper tree structure to receive the object
\kbd{x}. For instance, let \kbd{y} the zero integer with length equal to 2;
then \kbd{y} is too small to accommodate any non-zero \typ{INT}. In general
common sense tells you what is possible, keeping in mind the PARI
philosophy which says that if it makes sense it is valid. For instance, the
assignment of an imprecise object into a precise one does \emph{not} make
sense. However, a change in precision of imprecise objects is allowed.

All functions ending in ``\kbd{z}'' such as \teb{gaddz}
(see~\secref{se:low_level}) implicitly use this function. In fact what they
exactly do is record {\teb{avma}} (see~\secref{se:garbage}), perform the
required operation, \teb{gaffect} the result to the last operand, then
restore the initial \kbd{avma}.

You can assign ordinary C long integers into a PARI object (not necessarily
of type \typ{INT}). Use the function \teb{gaffsg} with the following
syntax:

\fun{void}{gaffsg}{long s, GEN y}

\misctitle{Note}: due to the requirements mentioned above, it is usually
a bad idea to use \tet{gaffect} statements. There is one exception: for simple
objects (e.g.~leaves) whose size is controlled, they can be easier to use than
gerepile, and about as efficient.

\misctitle{Coercion}. It is often useful to coerce an inexact object to a
given precision. For instance at the beginning of a routine where precision
can be kept to a minimum; otherwise the precision of the input is used in all
subsequent computations, which is inefficient if the latter is known to
thousands of digits. One may use the \kbd{gaffect} function for this, but it
is easier and more efficient to call

\fun{GEN}{gtofp}{GEN x, long prec} converts the complex number~\kbd{x}
(\typ{INT}, \typ{REAL}, \typ{FRAC}, \typ{QUAD} or \typ{COMPLEX}) to either
a \typ{REAL} or \typ{COMPLEX} whose components are \typ{REAL} of length
\kbd{prec}.

\subsec{Copy}. It is also very useful to \idx{copy} a PARI object, not
just by moving around a pointer as in the \kbd{y = x} example, but by
creating a copy of the whole tree structure, without pre-allocating a
possibly complicated \kbd{y} to use with \kbd{gaffect}. The function which
does this is called \teb{gcopy}. Its syntax is:

\fun{GEN}{gcopy}{GEN x}

\noindent and the effect is to create a new copy of x on the PARI stack.

Sometimes, on the contrary, a quick copy of the skeleton of \kbd{x} is
enough, leaving pointers to the original data in \kbd{x} for the sake of
speed instead of making a full recursive copy. Use
\fun{GEN}{shallowcopy}{GEN x} for this. Note that the result is not suitable
for \tet{gerepileupto} !

Make sure at this point that you understand the difference between \kbd{y =
x}, \kbd{y = gcopy(x)}, \kbd{y = shallowcopy(x)} and \kbd{gaffect(x,y)}.

\subsec{Clones}.\sidx{clone}\label{se:clone}
Sometimes, it is more efficient to create a \emph{persistent} copy of a PARI
object. This is not created on the stack but on the heap, hence unaffected by
\tet{gerepile} and friends. The function which does this is called
\teb{gclone}. Its syntax is:

\fun{GEN}{gclone}{GEN x}

A clone can be removed from the heap (thus destroyed) using

\fun{void}{gunclone}{GEN x}

\noindent No PARI object should keep references to a clone which has been
destroyed!

\subsec{Conversions}.\sidx{conversions}
The following functions convert C objects to PARI objects (creating them on
the stack as usual):

\fun{GEN}{stoi}{long s}: C \kbd{long} integer  (``small'') to \typ{INT}.

\fun{GEN}{dbltor}{double s}: C \kbd{double} to \typ{REAL}. The accuracy of
the result is 19 decimal digits, i.e.~a type \typ{REAL} of length
\kbd{DEFAULTPREC}, although on 32-bit machines only 16 of them are
significant.

\noindent We also have the converse functions:

\fun{long}{itos}{GEN x}: \kbd{x} must be of type \typ{INT},

\fun{double}{rtodbl}{GEN x}: \kbd{x} must be of type \typ{REAL},

\noindent as well as the more general ones:

\fun{long}{gtolong}{GEN x},

\fun{double}{gtodouble}{GEN x}.

\section{Implementation of the PARI types}
\label{se:impl}

\noindent
We now go through each type and explain its implementation. Let \kbd{z} be a
\kbd{GEN}, pointing at a PARI object. In the following paragraphs, we will
constantly mix two points of view: on the one hand, \kbd{z} is treated as the
C pointer it is, on the other, as PARI's handle on some mathematical entity,
so we will shamelessly write $\kbd{z} \ne 0$ to indicate that the
\emph{value} thus represented is nonzero (in which case the
\emph{pointer}~\kbd{z} is certainly non-\kbd{NULL}). We offer no apologies
for this style. In fact, you had better feel comfortable juggling both views
simultaneously in your mind if you want to write correct PARI programs.

Common to all the types is the first codeword \kbd{z[0]}, which we do not
have to worry about since this is taken care of by \kbd{cgetg}. Its precise
structure depends on the machine you are using, but it always contain the
following data: the \emph{internal type number}\sidx{type number} associated
to the symbolic type name, the \emph{length} of the root in longwords, and a
technical bit which indicates whether the object is a clone or not (see
\secref{se:clone}). This last one is used by \kbd{gp} for internal garbage
collecting, you will not have to worry about it.

\noindent These data can be handled through the following \emph{macros}:

\fun{long}{typ}{GEN z} returns the type number of \kbd{z}.

\fun{void}{settyp}{GEN z, long n} sets the type number of \kbd{z} to
\kbd{n} (you should not have to use this function if you use \kbd{cgetg}).

\fun{long}{lg}{GEN z} returns the length (in longwords) of the root of \kbd{z}.

\fun{long}{setlg}{GEN z, long l} sets the length of \kbd{z} to \kbd{l} (you
should not have to use this function if you use \kbd{cgetg}; however, see
an advanced example in \secref{se:prog}).

\fun{long}{isclone}{GEN z} is \kbd{z} a clone?

\fun{void}{setisclone}{GEN z} sets the \emph{clone} bit.

\fun{void}{unsetisclone}{GEN z} unsets the \emph{clone} bit.

\misctitle{Remark.} The clone bit is there so that \kbd{gunclone} can check
it is deleting an object which was allocated by \kbd{gclone}. Miscellaneous
vector entries are often cloned by \kbd{gp} so that a GP statement like
\kbd{v[1] = x} does not involve copying the whole of \kbd{v}: the component
\kbd{v[1]} is deleted if its clone bit is set, and is replaced by a clone of
\kbd{x}. Don't set/unset yourself the clone bit unless you know what you are
doing: in particular \emph{never} set the clone bit of a vector component
when the said vector is scheduled to be uncloned. Hackish code may abuse the
clone bit to tag objects for reasons unrelated to the above instead of using
proper data structures. Don't do that.

These macros are written in such a way that you do not need to worry about
type casts when using them: i.e.~if \kbd{z} is a \kbd{GEN}, \kbd{typ(z[2])}
is accepted by your compiler, as well as the more proper \kbd{typ(gel(z,2))}.
Note that for the sake of efficiency, none of the codeword-handling macros
check the types of their arguments even when there are stringent restrictions
on their use.

Some types have a second codeword, used differently by each type, and
we will describe it as we now consider each of them in turn.

\subsec{Type \typ{INT} (integer)}:\sidx{integer}\kbdsidx{t_INT} this type has
a second codeword \kbd{z[1]} which contains the following information:

the sign of \kbd{z}: coded as $1$, $0$ or $-1$ if $\kbd{z} > 0$, $\kbd{z} = 0$,
$\kbd{z} < 0$ respectively.

the \emph{effective length} of \kbd{z}, i.e.~the total number of significant
longwords. This means the following: apart from the integer 0, every integer
is ``normalized'', meaning that the most significant mantissa longword is
non-zero. However, the integer may have been created with a longer length.
Hence the ``length'' which is in \kbd{z[0]} can be larger than the
``effective length'' which is in \kbd{z[1]}.

\noindent This information is handled using the following macros:

\fun{long}{signe}{GEN z} returns the sign of \kbd{z}.

\fun{void}{setsigne}{GEN z, long s} sets the sign of \kbd{z} to \kbd{s}.

\fun{long}{lgefint}{GEN z} returns the \idx{effective length} of \kbd{z}.

\fun{void}{setlgefint}{GEN z, long l} sets the effective length
of \kbd{z} to \kbd{l}.

The integer 0 can be recognized either by its sign being~0, or by its
effective length being equal to~2. Now assume that $\kbd{z} \ne 0$, and let
$$ |z| = \sum_{i = 0}^n z_i B^i,
  \quad\text{where}~z_n\ne 0~\text{and}~B = 2^{\kbd{BITS\_IN\_LONG}}.
$$
With these notations, $n$ is \kbd{lgefint(z) - 3}, and the mantissa of
$\kbd{z}$ may be manipulated via the following interface:

\fun{GEN}{int_MSW}{GEN z} returns a pointer to the most significant word of
\kbd{z}, $z_n$.

\fun{GEN}{int_LSW}{GEN z} returns a pointer to the least significant word of
\kbd{z}, $z_0$.

\fun{GEN}{int_W}{GEN z, long i} returns the $i$-th significant word of
\kbd{z}, $z_i$. Accessing the $i$-th significant word for $i > n$
yields unpredictable results.

\fun{GEN}{int_precW}{GEN z} returns the previous (less significant) word of
\kbd{z}, $z_{i-1}$ assuming \kbd{z} points to $z_i$.

\fun{GEN}{int_nextW}{GEN z} returns the next (more significant) word of \kbd{z},
$z_{i+1}$ assuming \kbd{z} points to $z_i$.

Unnormalized integers, such that $z_n$ is possibly $0$, are explicitly
forbidden. To enforce this, one may write an arbitrary mantissa then call

\fun{void}{int_normalize}{GEN z, long known0}

\noindent normalizes in place a non-negative integer (such that $z_n$ is
possibly $0$), assuming at least the first \kbd{known0} words are zero.

\noindent For instance a binary \kbd{and} could be implemented in the
following way:
\bprog
GEN AND(GEN x, GEN y) {
  long i, lx, ly, lout;
  long *xp, *yp, *outp; /* mantissa pointers */
  GEN out;

  if (!signe(x) || !signe(y)) return gen_0;
  lx = lgefint(x); xp = int_LSW(x);
  ly = lgefint(y); yp = int_LSW(y); lout = min(lx,ly); /* > 2 */

  out = cgeti(lout); out[1] = evalsigne(1) | evallgefint(lout);
  outp = int_LSW(out);
  for (i=2; i < lout; i++)
  {
    *outp = (*xp) & (*yp);
    outp  = int_nextW(outp);
    xp    = int_nextW(xp);
    yp    = int_nextW(yp);
  }
  if ( !*int_MSW(out) ) out = int_normalize(out, 1);
  return out;
}
@eprog

\noindent This low-level interface is mandatory in order to write portable
code since PARI can be compiled using various multiprecision kernels, for
instance the native one or GNU MP, with incompatible internal structures
(for one thing, the mantissa is oriented in different directions).

\noindent The following further macros are available:

\fun{long}{mpodd}{GEN x} which is 1 if \kbd{x} is odd, and 0 otherwise.

\fun{long}{mod2}{GEN x}, \fun{}{mod4}{x}, and so on up to \fun{}{mod64}{x},
which give the residue class of \kbd{x} modulo the corresponding power of
2, for \emph{positive}~\kbd{x}. By definition, $\kbd{mod}n(x) :=
\kbd{mod}n(|x|)$ for $x < 0$ (the macros disregard the sign), and the
result is undefined if $x = 0$.

These macros directly access the binary data and are thus much faster than
the generic modulo functions. Besides, they return long integers instead of
\kbd{GEN}s, so they do not clutter up the stack.

\subsec{Type \typ{REAL} (real number)}:\kbdsidx{t_REAL}\sidx{real number}
this type has a second codeword z[1] which also encodes its sign, obtained
or set using the same functions as for a \typ{INT}, and a binary exponent.
This exponent is handled using the following macros:

\fun{long}{expo}{GEN z} returns the exponent of \kbd{z}.
This is defined even when \kbd{z} is equal to zero, see
\secref{se:whatzero}.

\fun{void}{setexpo}{GEN z, long e} sets the exponent of \kbd{z} to \kbd{e}.

\noindent Note the functions:

\fun{long}{gexpo}{GEN z} which tries to return an exponent for \kbd{z},
even if \kbd{z} is not a real number.

\fun{long}{gsigne}{GEN z} which returns a sign for \kbd{z}, even when
\kbd{z} is neither real nor integer (a rational number for instance).

The real zero is characterized by having its sign equal to 0. If \kbd{z} is
not equal to~0, then is is represented as $2^e M$, where $e$ is the exponent,
and $M\in [1, 2[$ is the mantissa of $z$, whose digits are stored in
$\kbd{z[2]},\dots, \kbd{z[lg(z)-1]}$.

More precisely, let $m$ be the integer (\kbd{z[2]},\dots, \kbd{z[lg(z)-1]})
in base \kbd{2\pow BITS\_IN\_LONG}; here, \kbd{z[2]} is the most significant
longword and is normalized, i.e.~its most significant bit is~1. Then we have
$M := m \cdot 2^{1 - \kbd{bit\_accuracy(lg(z))}}$.

Thus, the real number $3.5$ to accuracy \kbd{bit\_accuracy(lg(z))} is
represented as \kbd{z[0]} (encoding $\kbd{type} = \typ{REAL}$, \kbd{lg(z)}),
\kbd{z[1]} (encoding $\kbd{sign} = 1$, $\kbd{expo} = 1$), $\kbd{z[2]} =
\kbd{0xe0000000}$, $\kbd{z[3]} =\dots = \kbd{z[lg(z)-1]} = \kbd{0x0}$.

\subsec{Type \typ{INTMOD}}:\kbdsidx{t_INTMOD}
\kbd{z[1]} points to the modulus, and \kbd{z[2]} at the number representing
the class \kbd{z}. Both are separate \kbd{GEN} objects, and both must be
\typ{INT}s, satisfying the inequality $0 \le \kbd{z[2]} < \kbd{z[1]}$.

It is good practice to keep the modulus object on the heap, so that new
\typ{INTMOD}s resulting from operations can point at this common object,
instead of carrying along their own copies of it on the stack. The library
functions implement this practice almost by default.

\subsec{Type \typ{FRAC} (rational number)}:%
\kbdsidx{t_FRAC}\sidx{rational number}
\kbd{z[1]} points to the numerator $n$, and \kbd{z[2]} to the denominator
$d$. Both must be of type \typ{INT} such that $d\neq 0$, $n > 0$ and
$(n,d) = 1$ (see \tet{gred_frac2}).

\subsec{Type \typ{COMPLEX} (complex number)}:%
\kbdsidx{t_COMPLEX}\sidx{complex number}
\kbd{z[1]} points to the real part, and \kbd{z[2]} to the imaginary part. A
priori \kbd{z[1]} and \kbd{z[2]} can be of any type, but only certain types
are useful and make sense (mostly \typ{INT}, \typ{REAL} and \typ{FRAC}).

\subsec{Type \typ{PADIC} ($p$-adic numbers)}:%
\sidx{p-adic number}\kbdsidx{t_PADIC} this type has a second codeword
\kbd{z[1]} which contains the following information: the $p$-adic precision
(the exponent of $p$ modulo which the $p$-adic unit corresponding to
\kbd{z} is defined if \kbd{z} is not~0), i.e.~one less than the number of
significant $p$-adic digits, and the exponent of \kbd{z}. This information
can be handled using the following functions:

\fun{long}{precp}{GEN z} returns the $p$-adic precision of \kbd{z}.

\fun{void}{setprecp}{GEN z, long l} sets the $p$-adic precision of \kbd{z}
to \kbd{l}.

\fun{long}{valp}{GEN z} returns the $p$-adic valuation of \kbd{z} (i.e. the
exponent). This is defined even if \kbd{z} is equal to~0, see
\secref{se:whatzero}.

\fun{void}{setvalp}{GEN z, long e} sets the $p$-adic valuation of \kbd{z}
to \kbd{e}.

In addition to this codeword, \kbd{z[2]} points to the prime $p$,
\kbd{z[3]} points to $p^{\text{precp(z)}}$, and \kbd{z[4]} points to
a\typ{INT} representing the $p$-adic unit associated to \kbd{z} modulo
\kbd{z[3]} (and to zero if \kbd{z} is zero). To summarize, if $z\neq
0$, we have the equality:
$$ \kbd{z} = p^{\text{valp(z)}} * (\kbd{z[4]} + O(\kbd{z[3]})),\quad
\text{where}\quad \kbd{z[3]} = O(p^{\text{precp(z)}}). $$

\subsec{Type \typ{QUAD} (quadratic number)}:\sidx{quadratic
number}\kbdsidx{t_QUAD} \kbd{z[1]} points to the canonical polynomial $P$
defining the quadratic field (as output by \tet{quadpoly}), \kbd{z[2]} to the
``real part'' and \kbd{z[3]} to the ``imaginary part''. The latter are of
type \typ{INT}, \typ{FRAC}, \typ{INTMOD}, or \typ{PADIC} and are to be taken
as the coefficients of \kbd{z} with respect to the canonical basis $(1,X)$ or
$\Q[X]/(P(X))$, see~\secref{se:compquad}. Exact complex numbers may be
implemented as quadratics, but \typ{COMPLEX} is in general more versatile
(\typ{REAL} components are allowed) and more efficient.

Operations involving a \typ{QUAD} and \typ{COMPLEX} are implemented by
converting the \typ{QUAD} to a \typ{REAL} (or \typ{COMPLEX} with \typ{REAL}
components) to the accuracy of the \typ{COMPLEX}. As a consequence,
operations between \typ{QUAD} and \emph{exact} \typ{COMPLEX}s are not allowed.

\subsec{Type \typ{POLMOD} (polmod)}:\kbdsidx{t_POLMOD}\sidx{polmod}
as for \typ{INTMOD}s, \kbd{z[1]} points to the modulus, and \kbd{z[2]}
to a polynomial representing the class of~\kbd{z}. Both must be of type
\typ{POL} in the same variable, satisfying the inequality $\deg \kbd{z[2]}
< \deg \kbd{z[1]}$. However, \kbd{z[2]} is allowed to be a simplification
of such a polynomial, e.g a scalar. This is tricky considering the
hierarchical structure of the variables; in particular, a polynomial in
variable of \emph{lesser} priority (see \secref{se:priority}) than the
modulus variable is valid, since it is considered as the constant term of
a polynomial of degree 0 in the correct variable. On the other hand a
variable of \emph{greater} priority is not acceptable; see
\secref{se:priority} for the problems which may arise.

\subsec{Type \typ{POL} (polynomial)}:\kbdsidx{t_POL}\sidx{polynomial} this
type has a second codeword. It contains a ``\emph{sign}'': 0 if the
polynomial is equal to~0, and 1 if not (see however the important remark
below) and a \emph{variable number} (e.g.~0 for $x$, 1 for $y$, etc\dots).

\noindent These data can be handled with the following macros: \teb{signe}
and \teb{setsigne} as for \typ{INT} and \typ{REAL},

\fun{long}{varn}{GEN z} returns the variable number of the object \kbd{z},

\fun{void}{setvarn}{GEN z, long v} sets the variable number of \kbd{z} to
\kbd{v}.

The variable numbers encode the relative priorities of variables as discussed
in \secref{se:priority}. We will give more details in \secref{se:vars}. Note
also the function \fun{long}{gvar}{GEN z} which tries to return a
\idx{variable number} for \kbd{z}, even if \kbd{z} is not a polynomial or
power series. The variable number of a scalar type is set by definition equal
to \tet{BIGINT}, which has lower priority than any other variable number.

The components \kbd{z[2]}, \kbd{z[3]},\dots \kbd{z[lg(z)-1]} point to the
coefficients of the polynomial \emph{in ascending order}, with \kbd{z[2]}
being the constant term and so on.

For an object of type \typ{POL}, \tet{leading_term}, \tet{constant_term},
\tet{degpol} return a pointer to the leading term (with respect to the main
variable of course), constant term, and degree of the polynomial (with the
convention $\deg(0) = -1$). Applied to any other type, the result is
unspecified. Note that no copy is made on the pari stack so the returned
value is not safe for a basic \kbd{gerepile} call. Note that $\kbd{degpol(z)}
= \kbd{lg(z)} - 3$.

The leading term is not allowed to be an exact $0$ (\emph{unnormalized
polynomial}). To prevent this, one may use

\fun{GEN}{normalizepol}{GEN x} applied to an unnormalized \typ{POL}~\kbd{x}
(with all coefficients correctly set except that \kbd{leading\_term(x)} might
be zero), normalizes \kbd{x} correctly in place and returns~\kbd{x}. For
internal use.

\fun{long}{degree}{GEN x} returns the degree of \kbd{x} with respect to its
main variable even when \kbd{x} is not a polynomial (a rational function for
instance). By convention, the degree of $0$ is~$-1$.

\misctitle{Important remark}. A zero polynomial can be characterized by the
fact that its sign is~0. However, its length may be greater than 2, meaning
that all the coefficients of the polynomial are equal to zero, but the
leading term \kbd{z[lg(z)-1]} is an inexact zero. More precisely,
\kbd{gcmp0(x)} is true for all coefficients \kbd{x} of the polynomial,
an \kbd{isexactzero(x)} is false for the leading coefficient. The same
remark applies to \typ{SER}s.

\subsec{Type \typ{SER} (power series)}:\kbdsidx{t_SER}\sidx{power series}
This type also has a second codeword, which encodes a ``\emph{sign}'', i.e.~0
if the power series is 0, and 1 if not, a \emph{variable number} as for
polynomials, and an \emph{exponent}. This information can be handled with the
following functions: \teb{signe}, \teb{setsigne}, \teb{varn}, \teb{setvarn}
as for polynomials, and \teb{valp}, \teb{setvalp} for the exponent as for
$p$-adic numbers. Beware: do \emph{not} use \teb{expo} and \teb{setexpo} on
power series.

The coefficients \kbd{z[2]}, \kbd{z[3]},\dots \kbd{z[lg(z)-1]} point to
the coefficients of \kbd{z} in ascending order. As for polynomials
(see remark there), the sign of a \typ{SER} is $0$ if and only if the
leading coefficient of the series is an inexact $0$. (It cannot be an
exact $0$.)

Note that the exponent of a power series can be negative, i.e.~we are
then dealing with a Laurent series (with a finite number of negative
terms).

\subsec{Type \typ{RFRAC} (rational function)}:%
\kbdsidx{t_RFRAC}\sidx{rational function} \kbd{z[1]} points to the
numerator $n$,
and \kbd{z[2]} on the denominator $d$. The denominator must be of type \typ{POL},
with variable of higher priority than the numerator. The numerator
$n$ is not an exact $0$ and $(n,d) = 1$ (see \tet{gred_rfac2}).

\subsec{Type \typ{QFR} (indefinite binary quadratic form)}:%
\kbdsidx{t_QFR}\sidx{indefinite binary quadratic form} \kbd{z[1]},
\kbd{z[2]}, \kbd{z[3]} point to the three coefficients of the form and are of
type \typ{INT}. \kbd{z[4]} is Shanks's distance function, and must be of type
\typ{REAL}.

\subsec{Type \typ{QFI} (definite binary quadratic form)}:%
\kbdsidx{t_QFI}\sidx{definite binary quadratic form} \kbd{z[1]}, \kbd{z[2]},
\kbd{z[3]} point to the three coefficients of the form. All three are of type
\typ{INT}.

\subsec{Type \typ{VEC} and \typ{COL} (vector)}:%
\kbdsidx{t_VEC}\kbdsidx{t_COL}\sidx{row vector}\sidx{column vector}
\kbd{z[1]}, \kbd{z[2]},\dots \kbd{z[lg(z)-1]} point to the components of the
vector.

\subsec{Type \typ{MAT} (matrix)}:\kbdsidx{t_MAT}\sidx{matrix} \kbd{z[1]},
\kbd{z[2]},\dots \kbd{z[lg(z)-1]} point to the column vectors of \kbd{z},
i.e.~they must be of type \typ{COL} and of the same length.

\subsec{Type \typ{VECSMALL} (vector of small integers)}:\kbdsidx{t_VECSMALL}
\kbd{z[1]}, \kbd{z[2]},\dots \kbd{z[lg(z)-1]} are ordinary signed long
integers. This type is used instead of a \typ{VEC} of \typ{INT}s for
efficiency reasons, for instance to implement efficiently permutations,
polynomial arithmetic and linear algebra over small finite fields, etc.

\noindent The next two types were introduced for specific \kbd{gp} use, and
you are better off using the standard malloc'ed C constructs when programming
in library mode. We quote them for completeness, advising you not to use
them:

\subsec{Type \typ{LIST} (list)}:\kbdsidx{t_LIST}\sidx{list} This one has a
second codeword which contains an effective length (handled through
\teb{lgeflist}~/ \teb{setlgeflist}). \kbd{z[2]},\dots, \kbd{z[lgeflist(z)-1]}
contain the components of the list.

\subsec{Type \typ{STR} (character string)}:%
\kbdsidx{t_STR}\sidx{character string}

\fun{char *}{GSTR}{z} (= \kbd{(z+1)}) points to the first character of the
(\kbd{NULL}-terminated) string.

\misctitle{Implementation note}: for the types including an exponent (or a
valuation), we actually store a biased non-negative exponent (bit-ORing the
biased exponent to the codeword), obtained by adding a constant to the true
exponent: either \kbd{HIGHEXPOBIT} (for \typ{REAL}) or \kbd{HIGHVALPBIT} (for
\typ{PADIC} and \typ{SER}). Of course, this is encapsulated by the
exponent/valuation-handling macros and needs not concern the library user.

\section{PARI variables}\label{se:vars} \subsec{Multivariate
objects}\sidx{variable (priority)}

\noindent We now consider variables and formal computations, and give the
technical details corresponding to the general discussion in
\secref{se:priority}. As we have seen in \secref{se:impl}, the codewords for
types \typ{POL} and \typ{SER} encode a ``variable number''. This is an
integer, ranging from $0$ to \kbd{MAXVARN}. Relative priorities may be
ascertained using

\fun{int}{varncmp}{long v, long w}

\noindent which is $>0$, $=0$, $<0$ whenever $v$ has lower, resp.~same,
resp.~higher priority than $w$.

The way an object is considered in formal computations depends entirely on
its ``principal variable number'' which is given by the function

\fun{long}{gvar}{GEN z}

\noindent which returns a \idx{variable number} for \kbd{z}, even if \kbd{z}
is not a polynomial or power series. The variable number of a scalar type is
set by definition equal to \tet{BIGINT} which has lower priority than any
valid variable number. The variable number of a recursive type which is not a
polynomial or power series is the variable number with highest priority among
its components. But for polynomials and power series only the ``outermost''
number counts (we directly access $\tet{varn}(x)$ in the codewords): the
representation is not symmetrical at all.

Under \kbd{gp}, one needs not worry too much since the interpreter defines
the variables as it sees them\footnote{*}{ The first time a given identifier
is read by the GP parser (and is not immediately interpreted as a function) a
new variable is created, and it is assigned a strictly lower priority than
any variable in use at this point. On startup, before any user input has
taken place, 'x' is defined in this way and has initially maximal priority
(and variable number $0$).}
%
and do the right thing with the polynomials produced (however, have a look at
the remark in \secref{se:rempolmod}).

But in library mode, they are tricky objects if you intend to build
polynomials yourself (and not just let PARI functions produce them, which is
less efficient). For instance, it does not make sense to have a variable
number occur in the components of a polynomial whose main variable has a
lower priority, even though PARI cannot prevent you from doing it; see
\secref{se:priority} for a discussion of possible problems in a similar
situation.

\subsec{Creating variables} A basic difficulty is to ``create'' a variable.
As we have seen in \secref{se:intro4}, a number of objects is associated to
variable number~$v$. Here is the complete list: \tet{pol_1}$[v]$ and
\tet{pol_x}$[v]$, which you can use in library mode and which represent,
respectively, the monic monomials of degrees 0 and 1 in~$v$;
\teb{varentries}$[v]$, and \teb{polvar}$[v]$. The latter two are only
meaningful to \kbd{gp}, but they have to be set nevertheless. All of them
must be properly defined before you can use a given integer as a variable
number.

Initially, this is done for $0$ (the variable \kbd{x} under \kbd{gp}), and
\tet{MAXVARN}, which is there to address the need for a ``temporary'' new
variable in library mode and cannot be input under \kbd{gp}. No documented
library function can create from scratch an object involving \tet{MAXVARN}
(of course, if the operands originally involve \kbd{MAXVARN}, the function
abides). We call the latter type a ``temporary variable''. The regular
variables meant to be used in regular objects, are called ``user
variables\sidx{variable (user)}''.

\subsubsec{User variables}\sidx{variable (user)}: When the program starts,
\kbd{x} is the only user variable (number~$0$). To define new ones, use

\fun{long}{fetch_user_var}{char *$s$}

\noindent which inspects the user variable named $s$ (creating it if needed),
and returns its variable number.
\bprog
long v = fetch_user_var("y");
GEN gy = pol_x[v];
@eprog\noindent
This function raises an error if $s$ is already registered as a function
name. 

\misctitle{Caveat}: you can use \tet{gp_read_str}
(see~\secref{se:gp_read_str}) to execute a GP command and create GP
variables on the fly as needed:
\bprog
GEN gy = gp_read_str("'y"); /*@Ccom returns pol\_x[$v$], for some $v$ */
long v = varn(gy);
@eprog\noindent
But please note the quote \kbd{'y} in the above. Using \kbd{gp\_read\_str("y")}
might work, but is dangerous, especially when programming functions to 
be used under \kbd{gp}. The latter reads the value of \kbd{y}, as 
\emph{currently} known by the \kbd{gp} interpreter, possibly creating it
in the process. But if \kbd{y} has been modified by previous \kbd{gp}
commands (e.g \kbd {y = 1}), then the value of \kbd{gy} is not what you
expected it to be and corresponds instead to the current value of the
\kbd{gp} variable (e.g \kbd{gen\_1}).

\misctitle{Technical remark} If you are rewriting the gp interpreter, you
may use the lower level

\fun{entree *}{fetch_named_var}{char *$s$}

\noindent which returns an \kbd{entree*} suitable for inclusion in the
interpreter hashlists of symbols.

\subsubsec{Temporary variables}\sidx{variable (temporary)}:
\kbd{MAXVARN} is available, but is better left to pari internal functions
(some of which do not check that \kbd{MAXVARN} is free for them to use,
which can be considered a bug). You can create more temporary variables
using

\fun{long}{fetch_var}{}\label{se:fetch_var}

\noindent
This returns a variable number which is guaranteed to be unused by the
library at the time you get it and as long as you do not delete it (we will
see how to do that shortly). This has \emph{higher} priority than any
temporary variable produced so far (\kbd{MAXVARN} is assumed to be the first
such). This call updates all the aforementioned internal arrays. In
particular, after the statement \kbd{v = fetch\_var()}, you can use
\kbd{pol\_1[v]} and \kbd{pol\_x[v]}. The variables created in this way have no
identifier assigned to them though, and they is printed as
\kbd{\#<\text{number}>}, except for \kbd{MAXVARN} which is printed
as~\kbd{\#}. You can assign a name to a temporary variable, after creating
it, by calling the function

\fun{void}{name_var}{long n, char *s}

\noindent after which the output machinery will use the name \kbd{s} to
represent the variable number~\kbd{n}. The GP parser will \emph{not}
recognize it by that name, however, and calling this on a variable known
to~\kbd{gp} raises an error. Temporary variables are meant to be used as free
variables, and you should never assign values or functions to them as you
would do with variables under~\kbd{gp}. For that, you need a user variable.

All objects created by \kbd{fetch\_var} are on the heap and not on the stack,
thus they are not subject to standard garbage collecting (they are not
destroyed by a \kbd{gerepile} or \kbd{avma = ltop} statement). When you do
not need a variable number anymore, you can delete it using

\fun{long}{delete_var}{}

\noindent which deletes the \emph{latest} temporary variable created and
returns the variable number of the previous one (or simply returns 0 if you
try, in vain, to delete \kbd{MAXVARN}). Of course you should make sure that
the deleted variable does not appear anywhere in the objects you use later
on. Here is an example:

\bprog
  long first = fetch_var();
  long n1 = fetch_var();
  long n2 = fetch_var(); /*@Ccom prepare three variables for internal use */
  ...
  /*@Ccom delete all variables before leaving */
  do { num = delete_var(); } while (num && num <= first);
@eprog\noindent
The (dangerous) statement

\bprog
  while (delete_var()) /*@Ccom empty */;
@eprog\noindent
removes all temporary variables in use, except \kbd{MAXVARN} which cannot be
deleted.

\section{Input and output}

\noindent
Two important aspects have not yet been explained which are specific to
library mode: input and output of PARI objects.

\subsec{Input}.

\noindent
For \idx{input}, PARI provides you with one powerful high level function
which enables you to input your objects as if you were under \kbd{gp}. In fact,
it \emph{is} essentially the GP syntactical parser, hence you can use it not
only for input but for (most) computations that you can do under \kbd{gp}.
It has the following syntax:\label{se:gp_read_str}

\fun{GEN}{gp_read_str}{char *s}

\noindent
In fact this function starts by \emph{filtering} out all spaces and comments
in the input string. They it calls the underlying basic function, the GP
parser proper: \fun{GEN}{gp_read_str}{char *s}, which is slightly faster but
which you probably do not need.

To read a \kbd{GEN} from a file, you can use the simpler interface

\fun{GEN}{gp_read_stream}{FILE *file}

\noindent which reads a character string of arbitrary length from the stream
\kbd{file} (up to the first complete expression sequence), applies
\kbd{gp\_read\_str} to it, and returns the resulting \kbd{GEN}. This way, you
do not have to worry about allocating buffers to hold the string. To
interactively input an expression, use \kbd{gp\_read\_stream(stdin)}.

Finally, you can read in a whole file, as in GP's \tet{read} statement

\fun{GEN}{gp_read_file}{char *name}

\noindent As usual, the return value is that of the last non-empty expression
evaluated. Note that \kbd{gp}'s metacommands are not recognized.

Once in a while, it may be necessary to evaluate a GP expression sequence
involving a call to a function you have defined in~C. This is easy using
\teb{install} which allows you to manipulate quite an arbitrary function (GP
knows about pointers!). The syntax is

\fun{void}{install}{void *f, char *name, char *code}

\noindent where \kbd{f} is the (address of) the function (cast to the C type
\kbd{void*}), \kbd{name} is the name by which you want to access your
function from within your GP expressions, and \kbd{code} is a character
string describing the function call prototype (see~\secref{se:gp.interface}
for the precise description of prototype strings). In case the function
returns a \kbd{GEN}, it must satisfy \kbd{gerepileupto} assumptions (see
\secref{se:garbage}).

\subsec{Output}.

\noindent
For \idx{output}, there exist essentially three different functions (with
variants), corresponding to the three main \kbd{gp} output formats (as described in
\secref{se:output}), plus three extra ones, respectively devoted to
\TeX\ output, string output, and debugging.

\noindent $\bullet$ ``raw'' format, obtained by using the function
\teb{brute} with the following syntax:

\fun{void}{brute}{GEN obj, char x, long n}

\noindent
This prints the PARI object \kbd{obj} in \idx{format} \kbd{x0.n}, using the
notations from \secref{se:format}. Recall that here \kbd{x} is either
\kbd{'e'}, \kbd{'f'} or \kbd{'g'} corresponding to the three numerical output
formats, and \kbd{n} is the number of printed significant digits, and should
be set to $-1$ if all of them are wanted (these arguments only affect the
printing of real numbers). Usually one does not need that much flexibility,
and gets by with the function

\fun{void}{outbrute}{GEN obj}, which is equivalent to \kbd{brute(x,'g',-1)},

\noindent or even better, with

\fun{void}{output}{GEN obj} which is equivalent to \kbd{outbrute(obj)}
followed by a newline and a buffer flush. This is especially nice during
debugging. For instance using \kbd{dbx} or \kbd{gdb}, if \kbd{obj} is a
\kbd{GEN}, typing \kbd{print output(obj)} enables you to see the
content of \kbd{obj} (provided the optimizer has not put it into a
register, but it is rarely a good idea to debug optimized code).

\noindent $\bullet$ ``prettymatrix'' format: this format is identical to the
preceding one except for matrices. The relevant functions are:

\fun{void}{matbrute}{GEN obj, char x, long n}

\fun{void}{outmat}{GEN obj}, which is followed by a newline and a buffer flush.

\noindent $\bullet$ ``prettyprint'' format: the basic function has an
additional parameter \kbd{m}, corresponding to the (minimum) field width
used for printing integers:

\fun{void}{sor}{GEN obj, char x, long n, long m}

\noindent The simplified version is

\fun{void}{outbeaut}{GEN obj} which is equivalent to
\kbd{sor(obj,'g',-1,0)} followed by a newline and a buffer flush.

\noindent $\bullet$ The first extra format corresponds to the \teb{texprint}
GP function, and gives a \TeX\ output of the result. It is obtained by
using:

\fun{void}{texe}{GEN obj, char x, long n}

\noindent $\bullet$ The second one is the function \teb{GENtostr} which
converts a PARI \kbd{GEN} to an ASCII string. The syntax is

\fun{char*}{GENtostr}{GEN obj}, wich returns a \kbd{malloc}'ed character
string (which you should \kbd{free} after use).

\noindent $\bullet$ The third and final one outputs the \idx{hexadecimal tree}
corresponding to the \kbd{gp} metacommand \kbd{\b x} using the function

\fun{void}{voir}{GEN obj, long nb}, which only outputs the first
\kbd{nb} words corresponding to leaves (very handy when you have a look at
big recursive structures). If you set this parameter to $-1$ all significant
words are printed. This last type of output is only used for debugging purposes.

\misctitle{Remark}. Apart from \teb{GENtostr}, all PARI output is done on
the stream \teb{outfile}, which by default is initialized to \teb{stdout}. If
you want that your output be directed to another file, you should use the
function \fun{void}{switchout}{char *name} where \kbd{name} is a
character string giving the name of the file you are going to use. The
output is \emph{appended} at the end of the file. In order to close
the file, simply call \kbd{switchout(NULL)}.

Similarly, errors are sent to the stream \teb{errfile} (\teb{stderr}
by default), and input is done on the stream \teb{infile}, which you can change
using the function \teb{switchin} which is analogous to \teb{switchout}.

\misctitle{(Advanced) Remark}. All output is done according to the values
of the \teb{pariOut}~/ \teb{pariErr} global variables which are pointers to
structs of pointer to functions. If you really intend to use these, this
probably means you are rewriting \kbd{gp}. In that case, have a look at the code in
\kbd{language/es.c} (\kbd{init80()} or \kbd{GENtostr()} for instance).

\subsec{Errors}.\sidx{error}\kbdsidx{talker}

\noindent
If you want your functions to issue error messages, you can use the general
error handling routine \tet{pari_err}. The basic syntax is
%
\bprog
  pari_err(talker, "error message");
@eprog\noindent
This prints the corresponding error message and exit the program (in
library mode; go back to the \kbd{gp} prompt otherwise).\label{se:err} You can
also use it in the more versatile guise
\bprog
  pari_err(talker, format, ...);
@eprog\noindent
where \kbd{format} describes the format to use to write the remaining
operands, as in the \teb{printf} function (however, see the next section).
The simple syntax above is just a special case with a constant format and no
remaining arguments.

\noindent
The general syntax is

\fun{void}{pari_err}{numerr,...}

\noindent where \kbd{numerr} is a codeword which indicates what to do with
the remaining arguments and what message to print. The list of valid keywords
is in \kbd{language/errmessages.c} together with the basic corresponding
message. For instance, \kbd{pari\_err(typeer,"extgcd")} prints the message:
\bprog
    ***   incorrect type in extgcd.
@eprog\noindent
To issue a warning, use

\fun{void}{pari_warn}{warnerr,...}
In that case, of course, we do \emph{not} abort the computation, just print
the requested message and go on. The basic example is
%
\bprog
    pari_warn(warner, "Strategy 1 failed. Trying strategy 2")
@eprog\noindent
which is the exact equivalent of \kbd{pari\_err(talker,...)} except that
you certainly do not want to stop the program at this point, just inform the
user that something important has occurred (in particular, this output would be
suitably highlighted under \kbd{gp}, whereas a simple \kbd{printf} would not).

The valid \emph{warning} keywords are \tet{warner} (general), \tet{warnprec}
(increasing precision), \tet{warnmem} (garbage collecting) and \tet{warnfile}
(error in file operation), used as follows:
\bprog
    pari_warn(warnprec, "bnfinit", newprec);
    pari_warn(warnmem,  "bnfinit");
    pari_warn(warnfile, "close", "log");  /* error when closing "log" */
@eprog

\subsec{Debugging output}.\sidx{debugging}\sidx{format}\label{se:dbg_output}

\noindent
The global variables \teb{DEBUGLEVEL} and \teb{DEBUGMEM} (corresponding
to the default \teb{debug} and \teb{debugmem}, see \secref{se:defaults})
are used throughout the PARI code to govern the amount of diagnostic and
debugging output, depending on their values. You can use them to debug your
own functions, especially after having made them accessible under \kbd{gp} through
the command \teb{install} (see \secref{se:install}).

For debugging output, you can use \kbd{printf} and the standard output
functions (\teb{brute} or \teb{output} mainly), but also some special purpose
functions which embody both concepts, the main one being

\fun{void}{fprintferr}{char *pariformat, ...}

\noindent
Now let us define what a PARI format is. It is a character string, similar
to the one \kbd{printf} uses, where \kbd{\%} characters have a special
meaning. It describes the format to use when printing the remaining operands.
But, in addition to the standard format types, you can use \kbd{\%Z} to
denote a \kbd{GEN} object (we would have liked to pick \kbd{\%G} but it was
already in use!). For instance you could write:
\bprog
pari_err(talker, "x[%d] = %Z is not invertible!", i, x[i])
@eprog
\noindent since the \tet{pari_err} function accepts PARI formats. Here \kbd{i}
is an \kbd{int}, \kbd{x} a \kbd{GEN} which is not a leaf and this would
insert in raw format the value of the \kbd{GEN} \kbd{x[i]}.

\subsec{Timers and timing output}.

\noindent
To profile your functions, you can use the PARI timer. The functions
\fun{long}{timer}{} and \fun{long}{timer2}{} return the elapsed time since
the last call of the same function (in milliseconds). Two different
functions (identical except for their independent time-of-last-call
memories!) are provided so you can have both global timing and fine tuned
profiling.

You can also use \fun{void}{msgtimer}{char *format,...}, which prints
prints \kbd{Time}, then the remaining arguments as specified by
\kbd{format} (which is a PARI format), then the output of \kbd{timer2}.

This mechanism is simple to use but not foolproof. If some other function
uses these timers, and many PARI functions do use \kbd{timer2} when
\tet{DEBUGLEVEL} is high enough, the timings will be meaningless. To handle
timing in a reentrant way, PARI defines a dedicated data type,
\tet{pari_timer}. The functions

\fun{void}{TIMERstart}{pari\_timer *T}

\fun{long}{TIMER}{pari\_timer *T}

\fun{long}{msgTIMER}{pari\_timer *T, char *format,...}

\noindent provide an equivalent to \kbd{timer} and \kbd{msgtimer}, except
they use a unique timer \kbd{T} containing all the information needed, so
that no other function can mess with your timings. They are used as follows:
\bprog
  pari_timer T;
  TIMERstart(&T); /* initialize timer */
  ...
  printf("Total time: %ld\n", TIMER(&T));
@eprog\noindent
or
\bprog
  pari_timer T;
  TIMERstart(&T);
  for (i = 1; i < 10; i++) {
    ...
    msgTIMER(&T, "for i = %ld (L[i] = %Z)", i, L[i]);
  }
@eprog

\section{A complete program}
\label{se:prog}

\noindent
Now that the preliminaries are out of the way, the best way to learn how to
use the library mode is to study a detailed example. We want to write a
program which computes the gcd of two integers, together with the Bezout
coefficients. We shall use the standard quadratic algorithm which is not
optimal but is not too far from the one used in the PARI function
\teb{bezout}.

Let $x,y$ two integers and initially
$ \pmatrix{s_x & s_y \cr t_x & t_y } = 
  \pmatrix{1 & 0 \cr 0 & 1}$, so that
$$ \pmatrix{s_x & s_y \cr
            t_x & t_y }
   \pmatrix{x \cr y } = 
   \pmatrix{x \cr y }.
$$
To apply the ordinary Euclidean algorithm to the right hand side,
multiply the system from the left by
$ \pmatrix{0 & 1 \cr 1 & -q }$,
with $q = \kbd{floor}(x / y)$. Iterate until $y = 0$ in the right hand side,
then the first line of the system reads
$$ s_x x + s_y y = \gcd(x,y).$$
In practice, there is no need to update $s_y$ and $t_y$ since
$\gcd(x,y)$ and $s_x$ are enough to recover $s_y$. The following program
is now straightforward. A couple of new functions appear in there, whose
description can be found in the technical reference manual in Chapter 5.

\bprogfile{../examples/extgcd.c}

Note that, for simplicity, the inner loop does not include any garbage
collection, hence memory use is quadratic in the size of the inputs instead
of linear.

\section{Adding functions to PARI}
\subsec{Nota Bene}.
%
As mentioned in the \kbd{COPYING} file, modified versions of the PARI package
can be distributed under the conditions of the GNU General Public License. If
you do modify PARI, however, it is certainly for a good reason, hence we
would like to know about it, so that everyone can benefit from it. There is
then a good chance that your improvements are incorporated into the next
release.

We classify changes to PARI into four rough classes, where changes of the
first three types are almost certain to be accepted. The first type includes
all improvements to the documentation, in a broad sense. This includes
correcting typos or inacurracies of course, but also items which are not
really covered in this document, e.g.~if you happen to write a tutorial,
or pieces of code exemplifying fine points unduly omitted in the present
manual.

The second type is to expand or modify the configuration routines and skeleton
files (the \kbd{Configure} script and anything in the \kbd{config/}
subdirectory) so that compilation is possible (or easier, or more efficient)
on an operating system previously not catered for. This includes discovering
and removing idiosyncrasies in the code that would hinder its portability.

The third type is to modify existing (mathematical) code, either to correct
bugs, to add new functionalities to existing functions, or to improve their
efficiency.

Finally the last type is to add new functions to PARI. We explain here how
to do this, so that in particular the new function can be called from \kbd{gp}.

\subsec{The calling interface from \kbd{gp}, parser codes}.
\label{se:gp.interface}
A \idx{parser code} is a character string describing all the GP parser
needs to know about the function prototype. It contains a sequence of the
following atoms:

\settabs\+\indent&\kbd{Dxxx}\quad&\cr
\noindent $\bullet$ Syntax requirements, used by functions like
 \kbd{for}, \kbd{sum}, etc.:
%
\+& \kbd{=} & separator \kbd{=} required at this point (between two
arguments)\cr

\noindent$\bullet$ Mandatory arguments, appearing in the same order as the
input arguments they describe:
%
\+& \kbd{G} & \kbd{GEN}\cr
\+& \kbd{\&}& \kbd{*GEN}\cr
\+& \kbd{L} & long {\rm (we implicitly identify \kbd{int} with \kbd{long})}\cr
\+& \kbd{S} & symbol (i.e.~GP identifier name). Function expects a
\kbd{*entree}\cr
\+& \kbd{V} & variable (as \kbd{S}, but rejects symbols associated to
functions)\cr
\+& \kbd{n} & variable, expects a \idx{variable number} (a \kbd{long}, not an
\kbd{*entree})\cr
\+& \kbd{I} & string containing a sequence of GP statements (a \var{seq}), %
to be processed by \kbd{gp\_read\_str}\cr
\+&&(useful for control statements)\cr
\+& \kbd{E} & string containing a \emph{single} GP statement (an %
\var{expr}), to be processed by \kbd{readexpr}\cr
\+& \kbd{r} & raw input (treated as a string without quotes). Quoted %
 args are copied as strings\cr
\+&&\quad Stops at first unquoted \kbd{')'} or \kbd{','}. Special chars can
be quoted using '\kbd{\bs}'\cr
\+&&\quad Example: \kbd{aa"b\bs n)"c} yields the string \kbd{"aab\bs{n})c"}\cr
\+& \kbd{s} & expanded string. Example: \kbd{Pi"x"2} yields \kbd{"3.142x2"}\cr
\+&& Unquoted components can be of any PARI type (converted following current
output\cr
\+&& format)\cr

\noindent$\bullet$ Optional arguments:
%
\+& \kbd{s*} & any number of strings, possibly 0 (see \kbd{s})\cr
\+& \kbd{D\var{xxx}} &  argument has a default value\cr

The \kbd{s*} code is technical and you probably do not need it, but we give
its description for completeness. It reads all remaining arguments in
\tev{string context} (see \secref{se:strings}), and sends a
(\kbd{NULL}-terminated) list of \kbd{GEN*} pointing to these. The automatic
concatenation rules in string context are implemented so that adjacent strings
are read as different arguments, as if they had been comma-separated. For
instance, if the remaining argument sequence is: \kbd{"xx" 1, "yy"}, the
\kbd{s*} atom sends a \kbd{GEN *g = \obr \&a, \&b, \&c, NULL\cbr}, where
$a$, $b$, $c$ are \kbd{GEN}s of type \typ{STR} (content \kbd{"xx"}),
\typ{INT} (equal to $1$) and \typ{STR} (content \kbd{"yy"}).

The format to indicate a default value (atom starts with a \kbd{D}) is
``\kbd{D\var{value},\var{type},}'', where \var{type} is the code for any
mandatory atom (previous group), \var{value} is any valid GP expression
which is converted according to \var{type}, and the ending comma is
mandatory. For instance \kbd{D0,L,} stands for ``this optional argument is
converted to a \kbd{long}, and is \kbd{0} by default''. So if the
user-given argument reads \kbd{1 + 3} at this point, \kbd{4L} is sent to
the function; and \kbd{0L} if the argument is omitted. The following
special syntaxes are available:

\settabs\+\indent\indent&\kbd{Dxxx}\quad& optional \kbd{*GEN},&\cr
\+&\kbd{DG}& optional \kbd{GEN}, & send \kbd{NULL} if argument omitted.\cr

\+&\kbd{D\&}& optional \kbd{*GEN}, send \kbd{NULL} if argument omitted.\cr

\+&\kbd{DV}& optional \kbd{*entree}, send \kbd{NULL} if argument omitted.\cr

\+&\kbd{DI}& optional \kbd{*char}, send \kbd{NULL} if argument omitted.\cr

\+&\kbd{Dn}& optional variable number, $-1$ if omitted.\cr

\noindent$\bullet$ Automatic arguments:
%
\+& \kbd{f} &  Fake \kbd{*long}. C function requires a pointer but we
do not use the resulting \kbd{long}\cr
\+& \kbd{p} &  real precision (default \kbd{realprecision})\cr
\+& \kbd{P} &  series precision (default \kbd{seriesprecision},
 global variable \kbd{precdl} for the library)\cr

\noindent $\bullet$ Return type: \kbd{GEN} by default, otherwise the
following can appear at the start of the code string:
%
\+& \kbd{i} & return \kbd{int}\cr
\+& \kbd{l} & return \kbd{long}\cr
\+& \kbd{v} & return \kbd{void}\cr

No more than 8 arguments can be given (syntax requirements and return types
are not considered as arguments). This is currently hardcoded but can
trivially be changed by modifying the definition of \kbd{argvec} in
\kbd{anal.c:identifier()}. This limitation should disappear in future
versions.

When the function is called under \kbd{gp}, the prototype is scanned and each time
an atom corresponding to a mandatory argument is met, a user-given argument
is read (\kbd{gp} outputs an error message it the argument was missing). Each time
an optional atom is met, a default value is inserted if the user omits the
argument. The ``automatic'' atoms fill in the argument list transparently,
supplying the current value of the corresponding variable (or a dummy
pointer).

For instance, here is how you would code the following prototypes, which
do not involve default values:
\bprog
GEN name(GEN x, GEN y, long prec)   ----> "GGp"
void name(GEN x, GEN y, long prec)  ----> "vGGp"
void name(GEN x, long y, long prec) ----> "vGLp"
long name(GEN x)                    ----> "lG"
int name(long x)                    ----> "iL"
@eprog\noindent
If you want more examples, \kbd{gp} gives you easy access to the parser codes
associated to all GP functions: just type \kbd{\b{h} \var{function}}. You
can then compare with the C prototypes as they stand in the
\kbd{paridecl.h}.

\misctitle{Remark}: If you need to implement complicated control statements
(probably for some improved summation functions), you need to know about the
\teb{entree} type, which is not documented. Check the comment
at the end of \kbd{language/init.c} and the source code in
\kbd{language/sumiter.c}.
\smallskip

\subsec{Coding guidelines}.
\noindent
Code your function in a file of its own, using as a guide other functions
in the PARI sources. One important thing to remember is to clean the stack
before exiting your main function, since otherwise successive calls to
the function clutters the stack with unnecessary garbage, and stack
overflow occurs sooner. Also, if it returns a \kbd{GEN} and you want it
to be accessible to \kbd{gp}, you have to make sure this \kbd{GEN} is
suitable for \kbd{gerepileupto} (see \secref{se:garbage}).

If error messages or warnings are to be generated in your function, use
\kbd{pari\_err} and \kbd{pari\_warn} respectively.
Recall that \kbd{pari\_err} does not return but ends with a \kbd{longjmp}
statement. As well, instead of explicit \kbd{printf}~/ \kbd{fprintf}
statements, use the following encapsulated variants:

\fun{void}{pariflush}{}: flush output stream.

\fun{void}{pariputc}{char c}: write character \kbd{c} to the output stream.

\fun{void}{pariputs}{char *s}: write \kbd{s} to the output stream.

\fun{void}{fprintferr}{char *s}: write \kbd{s} to the error stream
(this function is in fact much more versatile, see \secref{se:dbg_output}).

Declare all public functions in an appropriate header file, if you
want to access them from C. For example, if dynamic
loading is not available, you may need to modify PARI to access these
functions, so put them in \kbd{paridecl.h}. The other functions should
be declared \kbd{static} in your file.

Your function is now ready to be used in library mode after compilation and
creation of the library. If possible, compile it as a shared library (see
the \kbd{Makefile} coming with the \kbd{extgcd} example in the
distribution). It is however still inaccessible from \kbd{gp}.\smallskip

\subsec{Integration with \kbd{gp} as a shared module}

To tell \kbd{gp} about your function, you must do the following. First, find a
name for it. It does not have to match the one used in library mode, but
consistency is nice. It has to be a valid GP identifier, i.e.~use only
alphabetic characters, digits and the underscore character (\kbd{\_}), the
first character being alphabetic.

Then figure out the correct \idx{parser code} corresponding to the
function prototype, as explained above (\secref{se:gp.interface}).

Now, assuming your Operating System is supported by \tet{install},
write a GP script like the following:

\bprog
install(libname, code, gpname, library)
addhelp(gpname, "some help text")
@eprog
\noindent(see \secref{se:addhelp} and~\ref{se:install}). The \idx{addhelp}
part is not mandatory, but very useful if you want others to use your
module. \kbd{libname} is how the function is named in the library,
usually the same name as one visible from C.

Read that file from your \kbd{gp} session (from your \idx{preferences
file} for instance, see \secref{se:gprc}), and that's it. You can now use
the new function \var{gpname} under \kbd{gp}, and we would very much like
to hear about it!

\subsec{Integration the hard way}

If \tet{install} is not available, things are more complicated: you have
to hardcode your function in the \kbd{gp} binary (or install
\idx{Linux}). Here is what needs to be done:

You need to choose a section and add a file
\kbd{functions/\var{section}/\var{gpname}}
containing the following, keeping the notation above:
\bprog
Function:  @com\var{gpname}
Section:   @com\var{section}
C-Name:    @com\var{libname}
Prototype: @com\var{code}
Help:      @com\var{some help text}
@eprog\noindent
(If the help text does not fit on a single line, continuation lines must
start by a whitespace character.) A GP2C-related \kbd{Description} field
is also available to improve the code GP2C generates when compiling
scripts involving your function. See the GP2C documentation for details.

At this point you can recompile \kbd{gp}, which will first rebuild the
functions database.

\subsec{Example}.
%
A complete description could look like this:

\bprog
{
  install(bnfinit0, "GD0,L,DGp", ClassGroupInit, "libpari.so");
  addhelp(ClassGroupInit, "ClassGroupInit(P,{flag=0},{data=[]}):
    compute the necessary data for ...");
}
@eprog\noindent which means we have a function \kbd{ClassGroupInit} under
\kbd{gp}, which calls the library function \kbd{bnfinit0} . The function has
one mandatory argument, and possibly two more (two \kbd{'D'} in the code),
plus the current real precision. More precisely, the first argument is a
\kbd{GEN}, the second one is converted to a \kbd{long} using \kbd{itos}
(\kbd{0} is passed if it is omitted), and the third one is also a \kbd{GEN},
but we pass \kbd{NULL} if no argument was supplied by the user. This matches
the C prototype (from \kbd{paridecl.h}):
%
\bprog
  GEN bnfinit0(GEN P, long flag, GEN data, long prec)
@eprog

This function is in fact coded in \kbd{basemath/buch2.c}, and is in this case
completely identical to the GP function \kbd{bnfinit} but \kbd{gp} does not
need to know about this, only that it can be found somewhere in the shared
library \kbd{libpari.so}.

\misctitle{Important note}: You see in this example that it is the
function's responsibility to correctly interpret its operands: \kbd{data =
NULL} is interpreted \emph{by the function} as an empty vector. Note that
since \kbd{NULL} is never a valid \kbd{GEN} pointer, this trick always
enables you to distinguish between a default value and actual input: the
user could explicitly supply an empty vector!

\misctitle{Note}: If \kbd{install} is not available, we have to add a file

 \kbd{functions/number\_fields/ClassGroupInit}

\noindent containing the following:
\bprog
Function: ClassGroupInit
Section: number_fields
C-Name: bnfinit0
Prototype: GD0,L,DGp
Help: ClassGroupInit(P,{flag=0},{tech=[]}): this routine does @com\dots
@eprog
\vfill\eject
